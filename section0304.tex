\begin{frame}
  Teszteljük le a \texttt{print()} tagfüggvény kimenetét!
  \vfill
  \begin{tabular}{ll}
    \textbf{Függvény} & \textbf{Funkció}\\ \hline
    \texttt{CaptureStdout()} & Megkezdi az \texttt{stdout}-ra írt tartalom rögzítését\\
    \texttt{GetCapturedStdout()} & Lekérdezi a rögzített tartalmat és leállítja a rögzítést\\
    \texttt{CaptureStderr()} & Megkezdi az \texttt{stderr}-re írt tartalom rögzítését\\
    \texttt{GetCapturedStderr()} & Lekérdezi a rögzített tartalmat és leállítja a rögzítést\\
  \end{tabular}
  \vfill
  Belső tagfüggvények, használatuk \kiemel{nem javasolt} (\hiv{\href{%
https://chromium.googlesource.com/external/github.com/google/googletest/+/HEAD/googletest/include/gtest/internal/gtest-port.h}%
{googletest forráskód}}).
\end{frame}

\begin{frame}
  \begin{exampleblock}{\textattachfile{09/matrix09test.cpp}{09/matrix09.cpp} %
    (\textattachfile{09/matrix09.h}{09/matrix09.h}, %
     \textattachfile{09/CMakeLists.txt}{09/CMakeLists.txt})}
    \small
    \lstinputlisting[style=cpp,linerange={76-85},numbers=left,firstnumber=76]{09/matrix09test.cpp}
  \end{exampleblock}
\end{frame}

\begin{frame}[fragile]
  \begin{block}{\textattachfile{09/build/Testing/Temporary/LastTest.log}{LastTest.log}}
    \footnotesize
    \begin{verbatim}
...
[ RUN      ] MulTest.print
/home/wajzy/Dokumentumok/gknb_intm006/GKxB_INTM006/09/matrix09test.cpp:84: Failure
Expected equality of these values:
  expected
    Which is: 0x5576ac2320a2
  output.c_str()
    Which is: 0x7fff6fa4a800
[  FAILED  ] MulTest.print (0 ms)
...
\end{verbatim}
  \end{block}
  \vfill
  Probléma: a C-stílusú karakterláncok \kiemel{címeit} hasonlítja össze, nem az ott lévő tartalmat!
\end{frame}

\begin{frame}
  \begin{center}
    C-stílusú karakterláncokkal szemben támasztható követelmények
    \medskip\\
    \resizebox{\textwidth}{!}{
    \begin{tabular}{llp{0.3\textwidth}}
      \textbf{Végzetes hibákhoz} & \textbf{Nem végzetes hibákhoz} & \textbf{Követelmény}\\ \hline
      ASSERT\_STREQ(\emph{str1}, \emph{str2}); & EXPECT\_STREQ(\emph{str1}, \emph{str2}); & A két C-stílusú karakterlánc 
tartalma azonos\\
      ASSERT\_STRNE(\emph{str1}, \emph{str2}); & EXPECT\_STRNE(\emph{str1}, \emph{str2}); & A két C-stílusú karakterlánc 
tartalma eltérő\\
      ASSERT\_STRCASEEQ(\emph{str1}, \emph{str2}); & EXPECT\_STRCASEEQ(\emph{str1}, \emph{str2}); & A két C-stílusú 
karakterlánc tartalma a kis- és nagybetűk eltérésétől eltekintve azonos\\
      ASSERT\_STRCASENE(\emph{str1}, \emph{str2}); & EXPECT\_STRCASENE(\emph{str1}, \emph{str2}); & A két C-stílusú 
karakterlánc tartalma a kis- és nagybetűk eltérését figyelmen kívül hagyva is eltérő\\
    \end{tabular}
    }
  \end{center}
\end{frame}

\begin{frame}
  Javítsuk a tesztesetet és készítsünk további, hasonló tagfüggvényeket (tesztekkel)!
  \begin{exampleblock}{\textattachfile{10/matrix10test.cpp}{10/matrix10test.cpp} %
    (\textattachfile{10/CMakeLists.txt}{10/CMakeLists.txt})}
    \small
    \lstinputlisting[style=cpp,linerange={76-86},numbers=left,firstnumber=76]{10/matrix10test.cpp}
  \end{exampleblock}
\end{frame}

\begin{frame}
  \begin{exampleblock}{\textattachfile{10/matrix10.h}{10/matrix10.h}}
    \lstinputlisting[style=cpp,linerange={3-3},numbers=left,firstnumber=3]{10/matrix10.h}
    \lstinputlisting[style=cpp,linerange={7-7},numbers=left,firstnumber=7]{10/matrix10.h}
    \lstinputlisting[style=cpp,linerange={11-11},numbers=left,firstnumber=11]{10/matrix10.h}
    \lstinputlisting[style=cpp,linerange={16-18},numbers=left,firstnumber=16]{10/matrix10.h}
    \lstinputlisting[style=cpp,linerange={24-24},numbers=left,firstnumber=24]{10/matrix10.h}
  \end{exampleblock}
\end{frame}

\begin{frame}
  \begin{exampleblock}{\textattachfile{10/matrix10.h}{10/matrix10.h}}
    \scriptsize
    \lstinputlisting[style=cpp,linerange={36-51},numbers=left,firstnumber=36]{10/matrix10.h}
  \end{exampleblock}
\end{frame}

\begin{frame}
  \begin{exampleblock}{\textattachfile{10/matrix10test.cpp}{10/matrix10test.cpp}}
    \scriptsize
    \lstinputlisting[style=cpp,linerange={88-102},numbers=left,firstnumber=88]{10/matrix10test.cpp}
  \end{exampleblock}
\end{frame}
