\documentclass[aspectratio=169]{beamer}

\usepackage[utf8]{inputenc}
\usepackage{default}
\usepackage[T1]{fontenc}
\usepackage[magyar]{babel}
\usepackage{indentfirst}
\usepackage{listingsutf8}
\usepackage{graphicx}
\lstset{literate=
  {á}{{\'a}}1 {é}{{\'e}}1 {í}{{\'i}}1 {ó}{{\'o}}1 {ú}{{\'u}}1
  {Á}{{\'A}}1 {É}{{\'E}}1 {Í}{{\'I}}1 {Ó}{{\'O}}1 {Ú}{{\'U}}1
  {à}{{\`a}}1 {è}{{\`e}}1 {ì}{{\`i}}1 {ò}{{\`o}}1 {ù}{{\`u}}1
  {À}{{\`A}}1 {È}{{\'E}}1 {Ì}{{\`I}}1 {Ò}{{\`O}}1 {Ù}{{\`U}}1
  {ä}{{\"a}}1 {ë}{{\"e}}1 {ï}{{\"i}}1 {ö}{{\"o}}1 {ü}{{\"u}}1
  {Ä}{{\"A}}1 {Ë}{{\"E}}1 {Ï}{{\"I}}1 {Ö}{{\"O}}1 {Ü}{{\"U}}1
  {â}{{\^a}}1 {ê}{{\^e}}1 {î}{{\^i}}1 {ô}{{\^o}}1 {û}{{\^u}}1
  {Â}{{\^A}}1 {Ê}{{\^E}}1 {Î}{{\^I}}1 {Ô}{{\^O}}1 {Û}{{\^U}}1
  {œ}{{\oe}}1 {Œ}{{\OE}}1 {æ}{{\ae}}1 {Æ}{{\AE}}1 {ß}{{\ss}}1
  {ç}{{\c c}}1 {Ç}{{\c C}}1 {ø}{{\o}}1 {å}{{\r a}}1 {Å}{{\r A}}1
  {€}{{\EUR}}1 {£}{{\pounds}}1 {ő}{{\H{o}}}1
}
\lstdefinestyle{cpp}{
  language=[ISO]C++,
  showstringspaces=false,
  keywordstyle=\color{MidnightBlue}\bfseries,
  stringstyle=\color{DarkOrchid},
  commentstyle=\color{Brown},
  morecomment=[l][\color{OliveGreen}]{\#}
}
\usepackage{hyperref}
\usepackage{attachfile}
\usepackage{multirow}
\attachfilesetup{color={1.0 0.6 0.0},author={HFM},description={Kattintson duplán a minta %
megtekintéséhez!},icon=Paperclip}
\usetheme{Warsaw}
\definecolor{kiemelesszin}{rgb}{0.6,0.0,0.0}
\definecolor{kiemelesszinZ}{rgb}{0.0,0.6,0.0}
\definecolor{kiemelesszinN}{RGB}{196,127,0}
\definecolor{hivatkozasszin}{rgb}{0.0,0.0,0.75}
\newcommand{\kiemel}[1]{{\color{kiemelesszin}#1}}
\newcommand{\kiemelZ}[1]{{\color{kiemelesszinZ}#1}}
\newcommand{\kiemelN}[1]{{\color{kiemelesszinN}#1}}
\newcommand{\hiv}[1]{{\color{hivatkozasszin}#1}}
\frenchspacing
\usetheme{Warsaw}

\title[Modern szoftverfejlesztési eszközök - egységtesztek]{C++ programok egységtesztelése googletest segítségével}
\subtitle{(GKxB\_INTM006)}
\author{Dr. Hatwágner F. Miklós}
\institute{Széchenyi István Egyetem, Győr}

\begin{document}

%1
\begin{frame}[plain]
  \titlepage
\end{frame}

\section{Dióhéjban a tesztelésről}

\begin{frame}
  Tesztelés célja: a hibákat megtalálni üzembe helyezés előtt
  \vfill
  Tesztelés alapelvei
  \begin{enumerate}
    \item A tesztelés bizonyos hibák jelenlétét jelezheti (ha nem jelzi, az nem jelent automatikusan hibamentességet)
    \item Nem lehetséges kimerítő teszt (a hangsúly a magas kockázatú részeken van)
    \item Korai teszt (minél hamarabb találjuk meg a hibát, annál olcsóbb javítani)
    \item Hibák csoportosulása (azokra a modulokra/bemenetekre kell tesztelni, amelyre a legvalószínűbben hibás a szoftver)
    \item Féregirtó paradoxon (a tesztesetek halmazát időnként bővíteni kell, mert ugyanazokkal a tesztekkel nem fedhetünk fel 
több hibát)
    \item Körülmények (tesztelés alapossága függ a felhasználás helyétől, a rendelkezésre álló időtől, stb.)
    \item A hibátlan rendszer téveszméje (A megrendelő elsősorban az igényeinek megfelelő szoftvert szeretne, és csak 
másodsorban hibamenteset; verifikáció vs. validáció)
  \end{enumerate}
\end{frame}

\begin{frame}
  Tesztelési technikák
  \begin{description}[mm]
    \item[Fekete dobozos (black-box, specifikáció alapú)] \hfill\\ A tesztelő nem látja a forrást, de a specifikációt igen, és 
hozzáfér a futtatható szoftverhez. Összehasonlítjuk a bemenetekre adott kimeneteket az elvárt kimenetekkel.
    \item[Fehér dobozos (white-box, strukturális teszt)] \hfill\\ Kész struktúrákat tesztelünk, pl.:
    \begin{itemize}
      \item kódsorok,
      \item elágazások,
      \item metódusok,
      \item osztályok,
      \item funkciók,
      \item modulok.
    \end{itemize}
    Lefedettség: a struktúra hány \%-át tudjuk tesztelni a tesztesetekkel?\\
    Egységteszt (unit test): a metódusok struktúra tesztje.
  \end{description}
\end{frame}

\begin{frame}
  A tesztelés szintjei:
  \begin{enumerate}
    \item komponensteszt (egy komponens tesztelése)
    \begin{enumerate}
      \item egységteszt
      \item modulteszt
    \end{enumerate}
    \item integrációs teszt (kettő vagy több komponens együttműködése)
    \item rendszerteszt (minden komponens együtt)
    \item átvételi teszt (kész rendszer)
  \end{enumerate}
\end{frame}

\begin{frame}
  Kik végzik a tesztelést?
  \begin{itemize}
    \item[1-3] Fejlesztő cég
    \item[4] Felhasználók
  \end{itemize}
  Komponensteszt
  \begin{itemize}
    \item fehér dobozos teszt
    \item egységteszt
    \begin{itemize}
      \item bemenet $\to$ kimenet vizsgálata
      \item nem lehet mellékhatása
      \item regressziós teszt: módosítással elronthattunk valamit, ami eddig jó volt $\to$ megismételt egységtesztek
    \end{itemize}
    \item modulteszt
    \begin{itemize}
      \item nem funkcionális tulajdonságok: sebesség, memóriaszivárgás (memory leak), szűk keresztmetszetek (bottleneck)
    \end{itemize}
  \end{itemize}
\end{frame}

\begin{frame}
  Integrációs teszt
  \begin{itemize}
    \item Komponensek közötti interfészek ellenőrzése, pl.
    \begin{itemize}
      \item komponens - komponens (egy rendszer komponenseinek együttműködése)
      \item rendszer - rendszer (pl. OS és a fejlesztett rendszer között)
    \end{itemize}
    \item Jellemző hibaokok: komponenseket eltérő csapatok fejlesztik, elégtelen kommunikáció
    \item Kockázatok csökkentése: mielőbbi integrációs tesztekkel
  \end{itemize}
\end{frame}

\begin{frame}
  Rendszerteszt: a termék megfelel-e a
  \begin{itemize}
    \item követelmény specifikációnak,
    \item funkcionális specifikációnak,
    \item rendszertervnek.
  \end{itemize}
  Gyakran fekete dobozos, külső cég végzi (elfogulatlanság)\\
  Leendő futtatási környezet imitációja
\end{frame}

\begin{frame}
  Átvételi teszt, fajtái:
  \begin{itemize}
    \item alfa: kész termék tesztelése a fejlesztőnél, de nem általa (pl. segédprogramok)
    \item béta: szűk végfelhasználói csoport
    \item felhasználói átvételi teszt: minden felhasználó használja, de nem éles termelésben. Jellemző a környezetfüggő hibák 
megjelenése (pl. sebesség)
    \item üzemeltetői átvételi teszt: rendszergazdák végzik, biztonsági mentés, helyreállítás, stb. helyesen működnek-e
  \end{itemize}
\end{frame}

\section{Források}

\begin{frame}
  Tesztelésről általában:\\
  \hiv{\href{https://www.tankonyvtar.hu/hu/tartalom/tamop425/0046\_szoftverteszteles/index.html}%
  {Ficsor Lajos, Kovács László, Kusper Gábor, Krizsán Zoltán: Szoftvertesztelés}}\\
  \hiv{\href{https://hstqb.org/downloadarea/istqb-ctfl-syllabus-2018-magyar/}{ISTQB CTFL Syllabus 2018}}\\  
\end{frame}

\end{document}
