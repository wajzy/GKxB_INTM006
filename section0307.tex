\begin{frame}
  Egészítsük ki a konstruktort kivétel dobásával, ha az eredeti vektor sorai nem azonos elemszámúak!
  \begin{exampleblock}{\textattachfile{13/matrix13.h}{13/matrix13.h} %
    (\textattachfile{13/CMakeLists.txt}{13/CMakeLists.txt})}
    \scriptsize
    \lstinputlisting[style=cpp,linerange={6-16},numbers=left,firstnumber=6]{13/matrix13.h}
    \lstinputlisting[style=cpp,linerange={26-26},numbers=left,firstnumber=26]{13/matrix13.h}
  \end{exampleblock}
\end{frame}

\begin{frame}
  \begin{exampleblock}{\textattachfile{13/matrix13.h}{13/matrix13.h}}
    \scriptsize
    \lstinputlisting[style=cpp,linerange={41-56},numbers=left,firstnumber=41]{13/matrix13.h}
  \end{exampleblock}
\end{frame}

\begin{frame}
  Módosítsuk és egészítsük ki tesztünket!
  \begin{exampleblock}{\textattachfile{13/matrix13test.cpp}{13/matrix13test.cpp}}
    \scriptsize
    \lstinputlisting[style=cpp,linerange={20-32},numbers=left,firstnumber=20]{13/matrix13test.cpp}
  \end{exampleblock}
\end{frame}

\begin{frame}
  \begin{exampleblock}{\textattachfile{13/matrix13test.cpp}{13/matrix13test.cpp}}
    \footnotesize
    \lstinputlisting[style=cpp,linerange={33-45},numbers=left,firstnumber=33]{13/matrix13test.cpp}
  \end{exampleblock}
\end{frame}

\begin{frame}
  \begin{exampleblock}{\textattachfile{13/matrix13test.cpp}{13/matrix13test.cpp}}
    \lstinputlisting[style=cpp,linerange={47-55},numbers=left,firstnumber=47]{13/matrix13test.cpp}
  \end{exampleblock}
\end{frame}

\begin{frame}
  \begin{center}
    Kivételek kiváltásával szemben támasztható követelmények
    \medskip\\
    \resizebox{\textwidth}{!}{
    \begin{tabular}{llp{0.3\textwidth}}
      \textbf{Végzetes hibákhoz} & \textbf{Nem végzetes hibákhoz} & \textbf{Követelmény}\\ \hline
      ASSERT\_THROW(\emph{statement}, \emph{exception\_type}); & EXPECT\_THROW(\emph{statement}, \emph{exception\_type}); & 
\emph{statement} hatására \emph{exception\_type} kivételnek kell keletkeznie \\
      ASSERT\_ANY\_THROW(\emph{statement}); & EXPECT\_ANY\_THROW(\emph{statement}); & \emph{statement} hatására 
valamilyen kivételnek kell keletkeznie \\
      ASSERT\_NO\_THROW(\emph{statement}); & EXPECT\_NO\_THROW(\emph{statement}); & \emph{statement} hatására 
semmilyen kivételnek sem szabad keletkeznie \\
    \end{tabular}
    }
  \end{center}
  \vfill
  A kivétel objektumban tárolt adatok (üzenet, egyedi hibakód, stb.) ezzel a módszerrel nem ellenőrizhetőek.
\end{frame}

\begin{frame}
  A haláltesztek (Death Tests) azt ellenőrzik, hogy valamilyen körülmény hatására a program leáll-e. Egészítsük ki a 
konstruktort úgy, hogy negatív sor- vagy oszlopszám esetén 1 hibakóddal álljon le a program!
  \begin{exampleblock}{\textattachfile{14/matrix14.h}{14/matrix14.h} %
    (\textattachfile{14/CMakeLists.txt}{14/CMakeLists.txt})}
    \small
    \lstinputlisting[style=cpp,linerange={28-33},numbers=left,firstnumber=28]{14/matrix14.h}
  \end{exampleblock}
\end{frame}

\begin{frame}
  Ellenőrizzük, hogy a program valóban leáll-e az elvárt módon!
  \begin{exampleblock}{\textattachfile{14/matrix14test.cpp}{14/matrix14test.cpp}}
    \small
    \lstinputlisting[style=cpp,linerange={119-123},numbers=left,firstnumber=119]{14/matrix14test.cpp}
  \end{exampleblock}
\end{frame}

\begin{frame}
  \begin{center}
    Halálteszteket támogató makrók
    \medskip\\
    \resizebox{\textwidth}{!}{
    \begin{tabular}{llp{0.3\textwidth}}
      \textbf{Végzetes hibákhoz} & \textbf{Nem végzetes hibákhoz} & \textbf{Követelmény}\\ \hline
      ASSERT\_DEATH(\emph{statement}, \emph{matcher}); & EXPECT\_DEATH(\emph{statement}, \emph{matcher}); & \emph{statement} 
programleállást idéz elő \emph{matcher} üzenettel \\
      ASSERT\_DEATH\_IF\_SUPPORTED(\emph{statement}, \emph{matcher}); & EXPECT\_DEATH\_IF\_SUPPORTED(\emph{statement}, 
\emph{matcher}); & Csak akkor ellenőrzi, hogy \emph{statement} 
programleállást idéz-e elő \emph{matcher} üzenettel, ha a haláltesztek támogatottak \\
      ASSERT\_EXIT(\emph{statement}, \emph{predicate}, \emph{matcher}); & EXPECT\_EXIT(\emph{statement}, \emph{predicate}, 
\emph{matcher}); & \emph{statement} 
programleállást idéz elő \emph{matcher} üzenettel, a kilépési kódot \emph{predicate}-nek megfelelőre állítja \\
    \end{tabular}
    }
  \end{center}
\end{frame}

\begin{frame}
  Paraméterezés:
  \begin{description}[mm]
    \item[\emph{statement}] \hfill\\ A programleálláshoz vezető (egyszerű vagy összetett) utasítás.
    \item[\emph{predicate}] \hfill\\ Függvény vagy függvény objektum, ami \texttt{int} paramétert vár és \texttt{bool}-t 
szolgáltat:
    \begin{itemize}
      \item \texttt{::testing::ExitedWithCode(exit\_code)} \\ Az elvárt kilépési kódot ellenőrzi.
      \item \texttt{::testing::KilledBySignal(signal\_number)} \\ Ellenőrzi, hogy a programot az elvárt jelzés szakította-e 
félbe (Windows-on nem támogatott).
    \end{itemize}
  \end{description}
\end{frame}

\begin{frame}
  Paraméterezés folyt.:
  \begin{description}[mm]
    \item[\emph{matcher}] \hfill\\ A szabvány hibacsatornára írt, elvárt üzenet. Ellenőrizhető:
\hiv{\href{https://pubs.opengroup.org/onlinepubs/009695399/basedefs/xbd_chap09.html\#tag_09_04}%
{POSIX kiterjesztett reguláris kifejezéssel}}
  \end{description}
  Megjegyzések
  \begin{itemize}
    \item A 0 kilépési kóddal leálló programot nem tekintik ,,halott'' programnak. A leállítás általában 
\texttt{abort()}, \texttt{exit()} hívással vagy egy jelzéssel történik.
    \item A haláltesztek készletének neve \texttt{DeathTest}-re kell, hogy végződjön 
(\hiv{\href{https://google.github.io/googletest/advanced.html\#death-test-naming}{részletek}}). 
Szálbiztos környezet szükséges lehet.
  \end{itemize}
\end{frame}
