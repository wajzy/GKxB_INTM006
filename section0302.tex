\begin{frame}
  Készítsünk lebegőpontos számokból álló mátrixokat, majd teszteljük a szorzást ismét!
  \begin{exampleblock}{\textattachfile{04/matrix04test.cpp}{04/matrix04test.cpp} %
    (\textattachfile{04/matrix04.h}{04/matrix04.h}, %
     \textattachfile{04/CMakeLists.txt}{04/CMakeLists.txt})}
    \small
    \lstinputlisting[style=cpp,linerange={31-41},numbers=left,firstnumber=31]{04/matrix04test.cpp}
  \end{exampleblock}
\end{frame}

\begin{frame}
  \begin{exampleblock}{\textattachfile{04/matrix04test.cpp}{04/matrix04test.cpp}}
    \small
    \lstinputlisting[style=cpp,linerange={42-52},numbers=left,firstnumber=42]{04/matrix04test.cpp}
  \end{exampleblock}
\end{frame}

\begin{frame}[fragile]
  \begin{block}{Kimenet}
    \scriptsize
    \begin{verbatim}
...
[ RUN      ] MulTest.rounding
/home/wajzy/Dokumentumok/gknb_intm006/GKxB_INTM006/04/matrix04test.cpp:49: Failure
Value of: multiplied.get(row, col)
  Actual: 1.41421
Expected: expected[row][col]
Which is: 1.41421
...
/home/wajzy/Dokumentumok/gknb_intm006/GKxB_INTM006/04/matrix04test.cpp:49: Failure
Value of: multiplied.get(row, col)
  Actual: 0.333333
Expected: expected[row][col]
Which is: 0.333333
[  FAILED  ] MulTest.rounding (0 ms)
...
\end{verbatim}
  \end{block}
  A kerekítési hibák érzékelhetetlenek a kimeneten és a teszt sikertelen.
\end{frame}

\begin{frame}
  Próbálkozzunk a beépített, lebegőpontos számokat összehasonlító makrókkal!
  \begin{exampleblock}{\textattachfile{05/matrix05test.cpp}{05/matrix05test.cpp} %
    (\textattachfile{05/matrix05.h}{05/matrix05.h}, %
     \textattachfile{05/CMakeLists.txt}{05/CMakeLists.txt})}
    \footnotesize
    \lstinputlisting[style=cpp,linerange={47-52},numbers=left,firstnumber=47]{05/matrix05test.cpp}
  \end{exampleblock}
\end{frame}

\begin{frame}[fragile]
  \begin{block}{Kimenet}
    \scriptsize
    \begin{verbatim}
...
[ RUN      ] MulTest.rounding
/home/wajzy/Dokumentumok/gknb_intm006/GKxB_INTM006/05/matrix05test.cpp:50: Failure
Value of: multiplied.get(row, col)
  Actual: 1.4142135623730951
Expected: expected[row][col]
Which is: 1.414213562
...
/home/wajzy/Dokumentumok/gknb_intm006/GKxB_INTM006/05/matrix05test.cpp:50: Failure
Value of: multiplied.get(row, col)
  Actual: 0.33333333333333331
Expected: expected[row][col]
Which is: 0.33333333300000001
[  FAILED  ] MulTest.rounding (0 ms)
...
\end{verbatim}
  \end{block}
  Most már látszik, hogy az értékek közötti különbség nagyobb, mint %
  \hiv{\href{https://randomascii.wordpress.com/2012/02/25/comparing-floating-point-numbers-2012-edition/}{4 ULP}} %
  (Units in the Last Place), ezért tekinti őket a teszt különbözőnek.
\end{frame}

\begin{frame}
  Növeljük meg a számok közötti legnagyobb megengedett eltérést!
  \begin{exampleblock}{\textattachfile{06/matrix06test.cpp}{06/matrix06test.cpp} %
    (\textattachfile{06/matrix06.h}{06/matrix06.h}, %
     \textattachfile{06/CMakeLists.txt}{06/CMakeLists.txt})}
    \footnotesize
    \lstinputlisting[style=cpp,linerange={47-53},numbers=left,firstnumber=47]{06/matrix06test.cpp}
  \end{exampleblock}
\end{frame}

\begin{frame}[fragile]
  \begin{block}{Kimenet}
    \scriptsize
    \begin{verbatim}
[==========] Running 2 tests from 1 test case.
[----------] Global test environment set-up.
[----------] 2 tests from MulTest
[ RUN      ] MulTest.meaningful
[       OK ] MulTest.meaningful (0 ms)
[ RUN      ] MulTest.rounding
[       OK ] MulTest.rounding (0 ms)
[----------] 2 tests from MulTest (1 ms total)

[----------] Global test environment tear-down
[==========] 2 tests from 1 test case ran. (1 ms total)
[  PASSED  ] 2 tests.
\end{verbatim}
  \end{block}
\end{frame}

\begin{frame}
  \begin{center}
    Lebegőpontos számokkal szemben támasztható követelmények
    \medskip\\
    \resizebox{\textwidth}{!}{
    \begin{tabular}{llp{0.3\textwidth}}
      \textbf{Végzetes hibákhoz} & \textbf{Nem végzetes hibákhoz} & \textbf{Követelmény}\\ \hline
      ASSERT\_FLOAT\_EQ(\emph{val1}, \emph{val2}); & EXPECT\_FLOAT\_EQ(\emph{val1}, \emph{val2}); & \emph{float} típusú 
értékek 4 ULP-n belül\\
      ASSERT\_DOUBLE\_EQ(\emph{val1}, \emph{val2}); & EXPECT\_DOUBLE\_EQ(\emph{val1}, \emph{val2}); & \emph{double} típusú 
értékek 4 ULP-n belül\\
      ASSERT\_NEAR(\emph{val1}, \emph{val2}, \emph{abs\_error}); & EXPECT\_NEAR(\emph{val1}, \emph{val2}, \emph{abs\_error}); & 
a két érték különbségének abszolút értéke nem nagyobb \emph{abs\_error}-nál\\
    \end{tabular}
    }
  \end{center}
\end{frame}
