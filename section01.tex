\begin{frame}
  Tesztelés célja: a hibákat megtalálni üzembe helyezés előtt
  \vfill
  Tesztelés alapelvei
  \begin{enumerate}
    \item A tesztelés bizonyos hibák jelenlétét jelezheti (ha nem jelzi, az nem jelent automatikusan hibamentességet)
    \item Nem lehetséges kimerítő teszt (a hangsúly a magas kockázatú részeken van)
    \item Korai teszt (minél hamarabb találjuk meg a hibát, annál olcsóbb javítani)
    \item Hibák csoportosulása (azokra a modulokra/bemenetekre kell tesztelni, amelyre a legvalószínűbben hibás a szoftver)
    \item Féregirtó paradoxon (a tesztesetek halmazát időnként bővíteni kell, mert ugyanazokkal a tesztekkel nem fedhetünk fel 
    több hibát)
    \item Körülmények (tesztelés alapossága függ a felhasználás helyétől, a rendelkezésre álló időtől, stb.)
    \item A hibátlan rendszer téveszméje (A megrendelő elsősorban az igényeinek megfelelő szoftvert szeretne, és csak 
    másodsorban hibamenteset; verifikáció vs. validáció)
  \end{enumerate}
\end{frame}

\begin{frame}
  Tesztelési technikák
  \begin{description}[mm]
    \item[Fekete dobozos (black-box, specifikáció alapú)] \hfill\\ A tesztelő nem látja a forrást, de a specifikációt igen, és 
hozzáfér a futtatható szoftverhez. Összehasonlítjuk a bemenetekre adott kimeneteket az elvárt kimenetekkel.
    \item[Fehér dobozos (white-box, strukturális teszt)] \hfill\\ Kész struktúrákat tesztelünk, pl.:
    \begin{itemize}
      \item kódsorok,
      \item elágazások,
      \item metódusok,
      \item osztályok,
      \item funkciók,
      \item modulok.
    \end{itemize}
    Lefedettség: a struktúra hány \%-át tudjuk tesztelni a tesztesetekkel?\\
    Egységteszt (unit test): a metódusok struktúra tesztje.
  \end{description}
\end{frame}

\begin{frame}
  A tesztelés szintjei:
  \begin{enumerate}
    \item komponensteszt (egy komponens tesztelése)
    \begin{enumerate}
      \renewcommand{\theenumii}{\alph{enumii}}
      \item egységteszt
      \item modulteszt
    \end{enumerate}
    \item integrációs teszt (kettő vagy több komponens együttműködése)
    \item rendszerteszt (minden komponens együtt)
    \item átvételi teszt (kész rendszer)
  \end{enumerate}
\end{frame}

\begin{frame}
  Kik végzik a tesztelést?
  \begin{itemize}
    \item[1-3] Fejlesztő cég
    \item[4] Felhasználók
  \end{itemize}
  Komponensteszt
  \begin{itemize}
    \item fehér dobozos teszt
    \item egységteszt
    \begin{itemize}
      \item bemenet $\to$ kimenet vizsgálata
      \item nem lehet mellékhatása
      \item regressziós teszt: módosítással elronthattunk valamit, ami eddig jó volt $\to$ megismételt egységtesztek
    \end{itemize}
    \item modulteszt
    \begin{itemize}
      \item nem funkcionális tulajdonságok: sebesség, memóriaszivárgás (memory leak), szűk keresztmetszetek (bottleneck)
    \end{itemize}
  \end{itemize}
\end{frame}

\begin{frame}
  Integrációs teszt
  \begin{itemize}
    \item Komponensek közötti interfészek ellenőrzése, pl.
    \begin{itemize}
      \item komponens - komponens (egy rendszer komponenseinek együttműködése)
      \item rendszer - rendszer (pl. OS és a fejlesztett rendszer között)
    \end{itemize}
    \item Jellemző hibaokok: komponenseket eltérő csapatok fejlesztik, elégtelen kommunikáció
    \item Kockázatok csökkentése: mielőbbi integrációs tesztekkel
  \end{itemize}
\end{frame}

\begin{frame}
  Rendszerteszt: a termék megfelel-e a
  \begin{itemize}
    \item követelmény specifikációnak,
    \item funkcionális specifikációnak,
    \item rendszertervnek.
  \end{itemize}
  Gyakran fekete dobozos, külső cég végzi (elfogulatlanság)\\
  Leendő futtatási környezet imitációja
\end{frame}

\begin{frame}
  Átvételi teszt, fajtái:
  \begin{itemize}
    \item alfa: kész termék tesztelése a fejlesztőnél, de nem általa (pl. segédprogramok)
    \item béta: szűk végfelhasználói csoport
    \item felhasználói átvételi teszt: minden felhasználó használja, de nem éles termelésben. Jellemző a környezetfüggő hibák 
    megjelenése (pl. sebesség)
    \item üzemeltetői átvételi teszt: rendszergazdák végzik, biztonsági mentés, helyreállítás, stb. helyesen működnek-e
  \end{itemize}
\end{frame}
