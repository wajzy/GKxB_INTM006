\begin{frame}
  A googletest főbb tulajdonságai
  \begin{itemize}
    \item platformfüggetlen (Linux, Windows, Mac)
    \item független és megismételhető tesztek
    \item struktúrálható tesztek (teszt program $\to$ teszt csomag $\to$ teszteset)
    \item informatív
    \item leveszi a tesztelés technikai részének terhét a tesztelőről
    \item gyors (megosztott erőforrások)
    \item könnyen tanulható (xUnit architektúra)
  \end{itemize}
\end{frame}

\begin{frame}
  \begin{itemize}
    \item Fejlesztőkörnyezet (pl. VS Code, gcc) telepítés, ha még nincs\\
    \item \texttt{cmake} (forráskódok automatizált fordítása) telepítés (Ubuntu)
    \begin{description}[m]
      \item[\texttt{sudo apt install cmake}] modern disztribúciókon
      \item[\hiv{\href{https://askubuntu.com/questions/355565/how-do-i-install-the-latest-version-of-cmake-from-the-command-line}{Régi disztribúciókon, ahol a cmake verzió nem elég friss (3.14+)}}]
    \end{description}
  \end{itemize}
\end{frame}

\begin{frame}
  Feladat
  \begin{itemize}
    \item[] Készítsünk mátrixműveleteket megvalósító osztályt, ami elsőként egy mátrixszorzást valósít meg.
    \item[] Az $A[a_{i,j}]_{m\times{n}}$ és $B[b_{i,j}]_{n\times{p}}$mátrixok
szorzatán azt a $C[c_{i,j}]_{m\times{p}}$ mátrixot értjük, amelyre $c_{i,j} =
a_{i,1}\cdot{b}_{1,j} + a_{i,2}\cdot{b}_{2,j} + \dots + a_{i,n}\cdot{b}_{n,j} =
\sum_{k=1}^{n}a_{i,k}\cdot{b}_{k,j}$ 
  \end{itemize}
  \vfill
  Megoldás
  \begin{itemize}
    \item Hozzuk létre a projekt könyvtárát: \texttt{mkdir 01}
    \item Lépjünk bele: \texttt{cd 01}
    \item Hozzuk létre a szükséges forráskódokat: \texttt{matrix01.h} (mátrix osztály), \texttt{example01.cpp} (főprogram)
  \end{itemize}
\end{frame}

\begin{frame}
  \begin{exampleblock}{\textattachfile{01/matrix01.h}{01/matrix01.h}}
    \footnotesize
    \lstinputlisting[style=cpp,linerange={1-13},numbers=left,firstnumber=1]{01/matrix01.h}
  \end{exampleblock}
\end{frame}

\begin{frame}
  \begin{exampleblock}{\textattachfile{01/matrix01.h}{01/matrix01.h}}
    \lstinputlisting[style=cpp,linerange={14-19},numbers=left,firstnumber=14]{01/matrix01.h}
  \end{exampleblock}
\end{frame}

\begin{frame}
  \begin{exampleblock}{\textattachfile{01/matrix01.h}{01/matrix01.h}}
    \lstinputlisting[style=cpp,linerange={21-29},numbers=left,firstnumber=21]{01/matrix01.h}
  \end{exampleblock}
\end{frame}

\begin{frame}
  \begin{exampleblock}{\textattachfile{01/matrix01.h}{01/matrix01.h}}
    \footnotesize
    \lstinputlisting[style=cpp,linerange={31-43},numbers=left,firstnumber=31]{01/matrix01.h}
  \end{exampleblock}
\end{frame}

\begin{frame}
  \begin{exampleblock}{\textattachfile{01/matrix01.h}{01/matrix01.h}}
    \small
    \lstinputlisting[style=cpp,linerange={45-56},numbers=left,firstnumber=45]{01/matrix01.h}
  \end{exampleblock}
\end{frame}

\begin{frame}
  \begin{exampleblock}{\textattachfile{01/example01.cpp}{01/example01.cpp}}
    \small
    \lstinputlisting[style=cpp,linerange={1-13},numbers=left,firstnumber=1]{01/example01.cpp}
  \end{exampleblock}
\end{frame}

\begin{frame}[fragile]
  \begin{exampleblock}{\textattachfile{01/example01.cpp}{01/example01.cpp}}
    \small
    \lstinputlisting[style=cpp,linerange={14-21},numbers=left,firstnumber=14]{01/example01.cpp}
  \end{exampleblock}
  \begin{block}{Fordítás (\texttt{g++ -o example01 example01.cpp}), kimenet:}
    \vspace{-.5cm}
    \begin{verbatim}
50      50      50
90      90      90
130     130     130
\end{verbatim}
  \vspace{-.4cm}
  \end{block}
\end{frame}

\begin{frame}
  Készítsünk az \texttt{example01.cpp} alapján googletest alapú tesztprogramot!
  \begin{exampleblock}{\textattachfile{01/matrix01test.cpp}{01/matrix01test.cpp}}
    \footnotesize
    \lstinputlisting[style=cpp,linerange={1-12},numbers=left,firstnumber=1]{01/matrix01test.cpp}
  \end{exampleblock}
\end{frame}

\begin{frame}
  \begin{exampleblock}{\textattachfile{01/matrix01test.cpp}{01/matrix01test.cpp}}
    \lstinputlisting[style=cpp,linerange={13-20},numbers=left,firstnumber=13]{01/matrix01test.cpp}
  \end{exampleblock}
\end{frame}

\begin{frame}
  \begin{exampleblock}{\textattachfile{01/matrix01test.cpp}{01/matrix01test.cpp}}
    \small
    \lstinputlisting[style=cpp,linerange={21-28},numbers=left,firstnumber=21]{01/matrix01test.cpp}
  \end{exampleblock}
\end{frame}

\begin{frame}
  A buildelést már a \texttt{cmake} végzi, hozzuk létre a konfigurációját!
  \begin{exampleblock}{\textattachfile{01/CMakeLists.txt}{01/CMakeLists.txt}}
    \scriptsize
    \lstinputlisting[linerange={1-15},numbers=left,firstnumber=1]{01/CMakeLists.txt}
  \end{exampleblock}
\end{frame}

\begin{frame}
  \begin{exampleblock}{\textattachfile{01/CMakeLists.txt}{01/CMakeLists.txt}}
    \footnotesize
    \lstinputlisting[linerange={17-29},numbers=left,firstnumber=17]{01/CMakeLists.txt}
  \end{exampleblock}
\end{frame}
%
%
%
\begin{frame}[fragile]
  \begin{description}[m]
    \footnotesize
    \item[\texttt{cmake -S . -B build}] \hfill \\ Összeállító (build) környezet létrehozása.
    \item[\texttt{cmake --build build}] \hfill \\ Összeállítás indítása.
    \item[\texttt{cd build \&\& ctest}] \hfill \\ Tesztprogram indítása.
  \end{description}
  \begin{block}{Kimenet}
    \scriptsize
    \vspace{-.5cm}
    \begin{verbatim}
Test project /home/wajzy/Dokumentumok/gknb_intm006/GKxB_INTM006/01/build
    Start 1: MulTest.meaningful
1/1 Test #1: MulTest.meaningful ...............   Passed    0.01 sec

100% tests passed, 0 tests failed out of 1

Total Test time (real) =   0.01 sec
\end{verbatim}
    \vspace{-.4cm}
  \end{block}
\end{frame}

\begin{frame}
  \begin{description}[m]
    \item[Teszteset (test case)] \hfill \\ "A set of preconditions, inputs, actions (where applicable), %
      expected results and postconditions, developed based on test conditions."\\
      (meaningful, ld. \texttt{matrix01test.cpp} 5. sor)
    \item[Tesztkészlet (test suite)] \hfill \\ "A set of test cases or test procedures to be executed in a specific test cycle."\\
      (MulTest, ld. \texttt{matrix01test.cpp} 5. sor)
    \item[Tesztprogram (test program)] \hfill \\ Egy vagy több tesztkészletet foglal magába.
  \end{description}
  Sajnos a googletest nevezéktana következetlen: \medskip\\
  \begin{tabular}{ll}
    \textbf{googletest} & \textbf{ISTQB}\\\hline
    teszt (test) & teszteset\\
    teszteset (test case, test suite) & tesztkészlet
  \end{tabular}
\end{frame}

\begin{frame}
  \begin{description}[m]
    \item[Assertion ($\approx$ állítás, követelés)] Ellenőrizzük valamely elvárásunk teljesülését $\to$ siker (success), %
      nem végzetes hiba (nonfatal failure), végzetes hiba (fatal failure).\\
      Makrók:
      \begin{description}[m]
        \item[\texttt{EXPECT\_*}] nem végzetes hibát generál, ajánlott (több hiba jelezhető egyszerre)
        \item[\texttt{ASSERT\_*}] végzetes hibát generál, azonnal leállítja a tesztesetet (nincs értelme a folytatásnak; pl. ha két mátrix nem azonos méretű, nincs értelme az elemeiket összehasonlítgatni). \kiemel{Erőforrások felszabadítása, takarítás is elmarad!}
      \end{description}
  \end{description}
\end{frame}

\begin{frame}
  Rontsuk el a kódot! (,,Elfelejtjük'' összegezni a szorzatokat.)
  \begin{exampleblock}{\textattachfile{02/matrix02.h}{02/matrix02.h} %
    (\textattachfile{02/matrix02test.cpp}{02/matrix02test.cpp}, %
    \textattachfile{02/CMakeLists.txt}{02/CMakeLists.txt})}
    \lstinputlisting[style=cpp,linerange={45-51},numbers=left,firstnumber=45]{02/matrix02.h}
  \end{exampleblock}
\end{frame}

\begin{frame}[fragile]
  \begin{block}{Kimenet, \textattachfile{02/build/Testing/Temporary/LastTest.log}{LastTest.log}}
    \tiny
    \begin{verbatim}
Test project /home/wajzy/Dokumentumok/gknb_intm006/GKxB_INTM006/02/build
    Start 1: MulTest.meaningful
1/1 Test #1: MulTest.meaningful ...............***Failed    0.00 sec

0% tests passed, 1 tests failed out of 1

Total Test time (real) =   0.01 sec

The following tests FAILED:
    1 - MulTest.meaningful (Failed)
Errors while running CTest
Output from these tests are in: /home/wajzy/Dokumentumok/gknb_intm006/GKxB_INTM006/02/build/Testing/Temporary/LastTest.log
Use "--rerun-failed --output-on-failure" to re-run the failed cases verbosely.
\end{verbatim}
  \end{block}
\end{frame}

\begin{frame}
  Most rontsuk el másképp a kódot! (Túl nagy lesz az eredmény mátrix.)
  \begin{exampleblock}{\textattachfile{03/matrix03.h}{03/matrix03.h} %
    (\textattachfile{03/CMakeLists.txt}{03/CMakeLists.txt})}
    \lstinputlisting[style=cpp,linerange={40-44},numbers=left,firstnumber=40]{03/matrix03.h}
  \end{exampleblock}
\end{frame}

\begin{frame}
  \begin{exampleblock}{\textattachfile{03/matrix03test.cpp}{03/matrix03test.cpp}}
    \footnotesize
    \lstinputlisting[style=cpp,linerange={21-33},numbers=left,firstnumber=21]{03/matrix03test.cpp}
  \end{exampleblock}
\end{frame}

\begin{frame}[fragile]
  \begin{block}{\textattachfile{03/build/Testing/Temporary/LastTest.log}{LastTest.log}}
    \tiny
    \vspace{-.5cm}
    \begin{verbatim}
Note: Google Test filter = MulTest.meaningful
[==========] Running 1 test from 1 test suite.
[----------] Global test environment set-up.
[----------] 1 test from MulTest
[ RUN      ] MulTest.meaningful
/home/wajzy/Dokumentumok/gknb_intm006/GKxB_INTM006/03/matrix03test.cpp:21: Failure
Expected equality of these values:
  expected.size()
    Which is: 3
  multiplied.getRowCount()
    Which is: 6
A sorok szama elter! Elvart: 3, kapott: 6
[  FAILED  ] MulTest.meaningful (0 ms)
[----------] 1 test from MulTest (0 ms total)

[----------] Global test environment tear-down
[==========] 1 test from 1 test suite ran. (0 ms total)
[  PASSED  ] 0 tests.
[  FAILED  ] 1 test, listed below:
[  FAILED  ] MulTest.meaningful

  1 FAILED TEST
\end{verbatim}
    \vspace{-.4cm}
  \end{block}
  \begin{itemize}
    \scriptsize
    \item Az \texttt{ASSERT\_EQ} leállította a tesztesetet.
    \item Testreszabott hibaüzeneteket jelenítettünk meg.
  \end{itemize}
\end{frame}

\begin{frame}
  \begin{center}
    Elemi követelmények
    \medskip\\
    \begin{tabular}{lll}
      \textbf{Végzetes hibákhoz} & \textbf{Nem végzetes hibákhoz} & \textbf{Követelmény}\\ \hline
      ASSERT\_TRUE(\emph{feltétel}) & EXPECT\_TRUE(\emph{feltétel}) & \emph{feltétel} igaz értékű\\
      ASSERT\_FALSE(\emph{feltétel}) & EXPECT\_FALSE(\emph{feltétel}) & \emph{feltétel} hamis értékű
    \end{tabular}
  \end{center}
\end{frame}

\begin{frame}
  \begin{center}
    Relációs követelmények
    \medskip\\
    \begin{tabular}{lll}
      \textbf{Végzetes hibákhoz} & \textbf{Nem végzetes hibákhoz} & \textbf{Követelmény}\\ \hline
      ASSERT\_EQ(\emph{val1}, \emph{val2}); & EXPECT\_EQ(\emph{val1}, \emph{val2}); & \emph{val1} == \emph{val2}\\
      ASSERT\_NE(\emph{val1}, \emph{val2}); & EXPECT\_NE(\emph{val1}, \emph{val2}); & \emph{val1} != \emph{val2}\\
      ASSERT\_LT(\emph{val1}, \emph{val2}); & EXPECT\_LT(\emph{val1}, \emph{val2}); & \emph{val1} < \emph{val2}\\
      ASSERT\_LE(\emph{val1}, \emph{val2}); & EXPECT\_LE(\emph{val1}, \emph{val2}); & \emph{val1} <= \emph{val2}\\
      ASSERT\_GT(\emph{val1}, \emph{val2}); & EXPECT\_GT(\emph{val1}, \emph{val2}); & \emph{val1} > \emph{val2}\\
      ASSERT\_GE(\emph{val1}, \emph{val2}); & EXPECT\_GE(\emph{val1}, \emph{val2}); & \emph{val1} >= \emph{val2}
    \end{tabular}
  \end{center}
\end{frame}

\begin{frame}
  Megjegyzések
  \begin{itemize}
    \item A feltüntetett operátoroknak definiáltnak kell lenniük \emph{val1} és \emph{val2} között. Lehetőségeink:
    \begin{enumerate}
      \item Felültöltjük az operátorokat.
      \item Az \{ASSERT,EXPECT\}\_\{TRUE,FALSE\} makrókat használjuk, de ezek nem írják a kimenetre az elvárt/kapott 
értékeket.
    \end{enumerate}
    \item A paraméterek egyszer lesznek kiértékelve, de nem definiált sorrendben (mellékhatások).
    \item Az \{ASSERT,EXPECT\}\_EQ makrók mutatók esetén a címeket hasonlítja össze, nem az ott lévő tartalmat! C-stílusú 
karakterláncok kezeléséhez külön makrók léteznek. (\texttt{string} objektumokkal nincs gond.)
    \item C++11 szabványnak megfelelő fordító esetén \texttt{NULL} helyett \texttt{nullptr}-t használjunk 
(utóbbi nem konvertálható implicit módon \texttt{int}-té)!
    \item Lebegőpontos számok összehasonlításakor kerekítési hibák adódhatnak.
  \end{itemize}
\end{frame}
