\begin{frame}
  Vegyük észre, hogy a tesztünkben egyre többször ismétlődnek részek:
  \begin{exampleblock}{\textattachfile{10/matrix10test.cpp}{10/matrix10test.cpp}}
    \tiny
    \lstinputlisting[style=cpp,linerange={76-80},numbers=left,firstnumber=76]{10/matrix10test.cpp}
    \lstinputlisting[style=cpp,linerange={88-92},numbers=left,firstnumber=88]{10/matrix10test.cpp}
    \lstinputlisting[style=cpp,linerange={96-100},numbers=left,firstnumber=96]{10/matrix10test.cpp}
  \end{exampleblock}
\end{frame}

\newcounter{fixtures}

\begin{frame}
  Megoldás: teszt \kiemel{fixture}-ök ($\approx$alkatrész) használata
  \begin{enumerate}
    \item Származtassunk le egy osztályt a \texttt{::testing::Test}-ből! Ha az \texttt{Osztaly}-t szeretnénk tesztelni, legyen 
a neve \texttt{OsztalyTest}!
    \item Deklaráljuk a többször használt tagokat! Legyenek védettek, hogy a leszármazottakból is használhatók legyenek!
    \item A tagokat inicializáljuk az alapértelmezett konstruktorban vagy a (felüldefiniált) \texttt{SetUp()} tagfüggvényben!
    \item Ha szükséges, készítsünk destruktort vagy (felüldefiniált) \texttt{TearDown()} tagfüggvényt az 
erőforrások felszabadítására!
    \item Ha szükséges, írjunk függvényeket, amiket több teszteset is hívhat!
    \setcounter{fixtures}{\theenumi}
  \end{enumerate}
\end{frame}

\begin{frame}
  \begin{enumerate}
    \setcounter{enumi}{\thefixtures}
    \item A tesztesetek definiálásakor a \texttt{TEST} helyett használjuk a \texttt{TEST\_F} makrót!
    \item A tesztkészlet neve egyezzen meg a fixture osztály nevével (\texttt{OsztalyTest})!
  \end{enumerate}
  \vfill
  Megjegyzések
  \begin{itemize}
    \item Az osztálynak már a tesztesetek makrói előtt definiáltnak kell lennie.
    \item Könnyű elgépelni a \texttt{SetUp()} és \texttt{TearDown()} függvények neveit, használjuk az \texttt{override} 
kulcsszót (C++11)!
    \item Minden egyes tesztesethez új példány készül a fixture-ből (nem ,,interferálnak'' a tesztesetek), majd:\\ 
alapértelmezett konstruktor $\to$ \texttt{SetUp()} $\to$ \texttt{TEST\_F} $\to$ \texttt{TearDown()} $\to$ destruktor.
  \end{itemize}
\end{frame}

\begin{frame}
  Mikor és miért érdemes konstruktort/destruktort használni?
  \begin{itemize}
    \item A \texttt{const} minősítővel ellátott tagváltozó csak a konstruktort követő inicializátor listával inicializálható. Jó ötlet
    a véletlen módosítások meggátolására.
    \item Ha a fixture osztályból származtatunk, az ős(ök) konstruktorának/destruktorának hívása mindenképpen végbemegy a megfelelő sorrendben. A \texttt{SetUp()/TearDown()} esetében erre a programozónak kell ügyelnie.
  \end{itemize}
\end{frame}

\begin{frame}
  Mikor és miért érdemes a \texttt{SetUp()/TearDown()} függvényeket használni?
  \begin{itemize}
    \item A C++ nem engedi meg virtuális függvények hívását a konstruktorokban és destruktorokban, mert elvileg így meghívható lehetne egy inicializálatlan objektum metódusa, és ezt túl körülményes ellenőrizni. (Ha megengedi, akkor is csak az aktuális objektum metódusát hívja.)
    \item A konstruktorban/destruktorban nem használhatóak az ASSERT\_* makrók. Megoldás:
    \begin{enumerate}
      \item \texttt{SetUp()/TearDown()} használata
      \item Az egész tesztprogramot állítjuk le egy \texttt{abort()} hívással.
    \end{enumerate}
    \item Ha a leállási folyamat során kivételek kelethezhetnek, azt a destruktorban nem lehet megbízhatóan lekezelni (definiálatlan viselkedés, akár azonnali programleállással).
  \end{itemize}
\end{frame}

\begin{frame}
  \begin{exampleblock}{\textattachfile{11/matrix11test.cpp}{11/matrix11test.cpp} %
    (\textattachfile{11/CMakeLists.txt}{11/CMakeLists.txt}, \textattachfile{11/matrix11.h}{11/matrix11.h})}
    \footnotesize
    \lstinputlisting[style=cpp,linerange={6-18},numbers=left,firstnumber=6]{11/matrix11test.cpp}
  \end{exampleblock}
\end{frame}

\begin{frame}
  \begin{exampleblock}{\textattachfile{11/matrix11test.cpp}{11/matrix11test.cpp}}
    \scriptsize
    \lstinputlisting[style=cpp,linerange={90-104},numbers=left,firstnumber=90]{11/matrix11test.cpp}
  \end{exampleblock}
\end{frame}

\begin{frame}[fragile]
  \begin{block}{Kimenet}
    \scriptsize
    \vspace{-.5cm}
    \begin{verbatim}
wajzy@lenovo:~/Dokumentumok/gknb_intm006/GKxB_INTM006/11$ cd build/ && ctest 
Test project /home/wajzy/Dokumentumok/gknb_intm006/GKxB_INTM006/11/build
    Start 1: MulTest.meaningful
1/6 Test #1: MulTest.meaningful ...............   Passed    0.00 sec
    Start 2: MulTest.equality
2/6 Test #2: MulTest.equality .................   Passed    0.00 sec
    Start 3: MulTest.rounding
3/6 Test #3: MulTest.rounding .................   Passed    0.00 sec
    Start 4: MatrixTest.print
4/6 Test #4: MatrixTest.print .................   Passed    0.00 sec
    Start 5: MatrixTest.toString
5/6 Test #5: MatrixTest.toString ..............   Passed    0.00 sec
    Start 6: MatrixTest.toCString
6/6 Test #6: MatrixTest.toCString .............   Passed    0.00 sec

100% tests passed, 0 tests failed out of 6

Total Test time (real) =   0.01 sec
\end{verbatim}
    \vspace{-.4cm}
  \end{block}
\end{frame}
